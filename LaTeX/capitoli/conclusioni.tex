L'elaborato ha dimostrato la plausibilità di strumenti non dedicati alla fotopletismografia per un'acquisizione accettabile del battito cardiaco. Le implicazioni di questo lavoro hanno un'utilità notevole qualora si progetti un porting del software prodotto verso sistemi mobili e smartphones, i quali hanno (vi sono esempi concreti nei repository di applicazioni dei più noti sistemi operativi mobili) sufficiente capacità computazionale per il compito.
 Applicazioni in grado di essere utilizzate in ogni momento e con comunissime intensità luminose di ambiente potrebbero esser di grande aiuto per soccorsi d'emergenza, intensa attività cardiaca nel fitness o per normali check-up di salute quotidiani.

Scopo del lavoro non è stato quello di produrre codice ``state-of-Art'': raggiunta una sufficiente bontà dell'output, qualsiasi aspetto di efficienza computazionale viene in secondo piano, sebbene si sia cercato di governarlo mediante i parametri (vedere appendice B). Il sorgente pertanto {\em non} è assolutamente da prendere a esempio su {\em come} implementare efficacemente quanto teorizzato; sviluppi futuri possono considerare l'idea di una riscrittura pressoché totale del codice per una maggiore aderenza ai patterns e agli stili di buona programmazione dei quali si possono trovare ampi riferimenti in letteratura.