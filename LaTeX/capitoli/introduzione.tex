\section{Il battito cardiaco}
L'informazione rappresentata dal numero di battiti cardiaci in un minuto (in seguito riferito, per brevità, con il termine battito cardiaco, heart rate o l'acronimo HR) è sempre stata utilizzata dall'essere umano come primitivo sapere sulla salute generale di una persona.
Il metodo tradizionalmente usato per conoscere il battito cardiaco è l'esercitazione di una pressione con le dita sul polso del soggetto, sebbene qualsiasi parte del corpo abbia lo stesso valore di HR. Ci si affida in questo modo al solo proprio senso tattile.

L'auscultazione tramite stetoscopio in epoca moderna ha fornito alla scienza medica un sensore più affidabile per il problema in esame, sensore che però in certi casi non si dimostra ottimale al contesto nel quale si voglia effettuare la misura. Alcuni di questi appartengono nel settore clinico e riguardano persone soggette a fenomeni di aritmia o irregolarità. Altri casi, certo meno gravi, rientrano nell'ambito di attività di fitness o motorie, dove i sensori per la misurazione elencati si dimostrano veramente poco pratici e immediati.

L'interesse del laureando si è concentrato proprio su questo ambito, in particolare sull'utilizzo delle strumentazioni elettroniche come calcolatori, videocamere e smartphones, atte a realizzare un sistema di acquisizione del battito cardiaco per via fotopletismografica.
%-----------------------------------------------------------------------------------------
\section{Scopo della tesi}	% SCOPO TESI
Questo lavoro si propone di dimostrare come il laureando abbia saputo comprendere un fenomeno fisico e implementare un algoritmo che ne dia un modello sufficientemente fedele, avendo come traccia l'articolo {\em Advancements in noncontact, multiparameter physiological measurements using a webcam} \cite{POH11} di Ming-Zher Poh.

Ipotesi cardine dell'elaborato è che l'acquisizione dei segnali dai quali si stimerà il battito cardiaco avvenga mediante una videocamera commerciale di bassa-media gamma, come può essere una webcam da personal computer che supporti risoluzioni come la 640x480 e la 1280x720 pixel, comunemente diffuse. Questo vincolo giustifica il didascalico ``strumentazione non dedicata'' nel titolo della tesi.

La congettura di Poh è di ottenere i suddetti segnali attraverso la registrazione, debitamente illuminata da incandescenza, della regione facciale del soggetto; il laureando, dopo aver cercato di seguirne l'esempio ed aver ottenuto un segnale troppo pieno di rumore per poter apprezzare i risultati, ha configurato il proprio software per acquisire tali dati da un dito del soggetto, retroilluminato e posto molto vicino all'obiettivo fotografico. Le misure a questo punto sono state considerate soddisfacenti e verranno mostrate nel capitolo ''Risultati sperimentali''.

Non è assolutamente tra gli obiettivi del laureando la smentita o la denigrazione nemmeno parziale del lavoro del Dottor Poh: si asserisce semplicemente di non essere riusciti a ottenerne un'identica versione apprezzabile e, salvo il punto ora discusso, si è data prova del suo corretto funzionamento.

